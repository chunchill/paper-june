
%!TEX root = ../article.tex

% Introduction
\section{Introduction}

Cloud Platform like OpenStack is a large complex software system. It is composed of several components and networks,it is the critical infrastructure of other software application, which means other applications should run on it. So  it already be designed with high availability and reliability in order to prevent downtime that would lead to big loss of business and revenue.When downtime occurs, the system operation administrators and the service support staff should look into the system trace and user logs to pinpoint the root cause of the failure. The straightforward process is to explore the log files manually in nature, However, a large software system normally consists of numbers modules and components in a complex environment,with each component and module writing its debug or trace log messages to a specific location, which may be a dedicated file or a database. And we fully believe that a systematic logging mechanism should be naturally built-in for a welcome software, the system logs should always reflect the running state of the software,from this point of view,there should be no exception for cloud platform. Due to the frequent interactions and high coupling,the system will generate enormous log data including at least normal information, warning information,debugging information,error information. Manually exploring millions of logs to find the root cause of a system fault is painful and inefficient. Because it is easy to image that the manual method is to search the key words in the fault information in the log data,that sound like a needle-in-a-haystack problem\cite{rao2011identifying}. \\


\setlength{\parindent}{2em}Cloud Platform like OpenStack is a cloud operating system that controls large pools of compute, storage, and networking resources throughout a datacenter, which become increasingly complex and the components within the entire system also become diverse. Once some key parts failed,the whole system would be seriously impacted due to frequent interactions. Therefore an effective fault detection and diagnosis framework can help system administrators to locate the fault and identify the root cause,which plays an critical role in large software system management. In this paper,we propose a log mining based fault diagnosis framework which is used to identify the logs which could facilitate the troubleshooting automatically via data mining from the log repository and QA knowledge base repository. We collect the logs data from the OpenStack platform, classify and analyse the logs,finally we will build a log repository, meanwhile, we try to use web crawler technology to fetch the QA topics on discussing the solution to a failure occurs in the OpenStack Platform,which usually contain a certain number of threads that users hope to provide a solution to the problem. To be briefly, in this paper,we will propose a improved classification method for fault logs,using topic clustering technology to improve the accuracy. Combining with mining the QA website,we will build a knowledge base repository to facilitate the RCA (Root Cause Analysis) process.In this Framework, the ELK (Elastic,LogStash,Kibana) Stack is adopted. LogStash is a log collector, Elastic is index database to store the log information, and the Kibana part is a real-time dashboard which is also a supplementary measure for fault diagnosis. \\

\setlength{\parindent}{2em} The remaining of this paper is organized as follows:Section II discuss the background and  related work. Section III introduce the core methodology about log mining and data prediction related to this paper. Section IV describe the overview design and implementation of the proposed framework. Section V present the experiment result and the corresponding analysis of the result. Section VI concludes with a discussion of the results. and finally is the Acknowledgement Section. 


